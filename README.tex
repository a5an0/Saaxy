% Created 2012-04-17 Tue 23:21
\documentclass[11pt]{article}
\usepackage[utf8]{inputenc}
\usepackage[T1]{fontenc}
\usepackage{graphicx}
\usepackage{longtable}
\usepackage{float}
\usepackage{wrapfig}
\usepackage{soul}
\usepackage{amssymb}
\usepackage{hyperref}


\title{README}
\author{Konrad Scorciapino}
\date{17 April 2012}

\begin{document}

\maketitle

\setcounter{tocdepth}{3}
\tableofcontents
\vspace*{1cm}
\section{About}
\label{sec-1}

  
  Saaxy is just a way to use services with the dead-simple interface of 
  IRC bots. The name comes from SaaS + Sexy, but some system tools are 
  included, as well, and it's dead-simple to add your own utilities.

\href{http://i.imgur.com/G89dU.png}{http://i.imgur.com/G89dU.png}

\section{Features}
\label{sec-2}


\subsection{Services}
\label{sec-2.1}



\begin{center}
\begin{tabular}{lll}
 Type          &  Name                                   &  How to use                                         \\
\hline
 Web           &  Freebase Search                        &  !freebase milton friedman                          \\
               &  Reddit - Top stories from a subreddit  &  !reddit paleo                                      \\
               &  Duckduckgo                             &  !ddg quid est veritas                              \\
               &  Wolfram Alpha                          &  !wa carbs in 300g of heavy cream                   \\
               &  Transmission - Torrent Listing         &  !trans                                             \\
\hline
 System        &  Festival speech synthesizer            &  !say cabbage-brained                               \\
               &  Clean history                          &  !clean                                             \\
               &  Eval and pretty-print                  &  !pp `((1 2) (3 4))                                 \\
               &  Exit Saaxy                             &  !bye                                               \\
               &  Eval expression                        &  !eval (+ 3 4)                                      \\
               &  Run shell command                      &  !sh date                                           \\
               &  \textbf{The help system}               &  !help                                              \\
\hline
 Portuguese    &  Rhyme dictionary in portuguese         &  !vadio coito 4                                     \\
               &  `Porto' dictionary                     &  !por amável                                        \\
               &  `Michaelis' dictionary                 &  !mic amável                                        \\
               &  `Aurélio' dictionary                   &  !au amável                                         \\
\hline
 Shortening    &  TinyURL shortening                     &  !tiny \href{http://goatse.cz}{http://goatse.cz}    \\
               &  TinyCC shortening                      &  !tinycc \href{http://goatse.cz}{http://goatse.cz}  \\
\hline
 Productivity  &  Current time                           &  !now                                               \\
               &  Tomatinho's history view               &  !th                                                \\
               &  Tomatinho's tube view                  &  !tt                                                \\
\hline
 English       &  Wordnet                                &  !dict revigorating                                 \\
\hline
 Adult         &  Monica's tube                          &  !monica bikini                                     \\
\hline
 Latin         &  Latin->English                         &  !latin veritas                                     \\
               &  English->Latin                         &  !tolat truth                                       \\
\end{tabular}
\end{center}



\subsection{Context}
\label{sec-2.2}


   Suppose you are reading a book. Instead of having to type everytime
   \emph{!dict scoundrel}, \emph{!dict baritone} etc, you can just specify that
   these commands are all relative to \emph{dict} with \emph{@}:


\begin{verbatim}
> @dict dict dict> scoundrel ...  dict> baritone ...  dict> @@
Context reset.  >
\end{verbatim}



\subsection{Linked items}
\label{sec-2.3}


   Some commands do not only list the results, but specify actions on
   them, acessible with ``\#''. For example:


\begin{verbatim}
➤ !ddg richard stallman Running...  ➤
0. https://pt.wikipedia.org/wiki/Richard_Matthew_Stallman Richard
        Matthew Stallman – Wikipédia, a enciclopédia livre
1. https://www.facebook.com/people/Richard-Stallman/100000599450702
        Facebook - Richard Stallman
2. http://www.stallman.org/ Personal home page ...

➤ #
0. https://pt.wikipedia.org/wiki/Richard_Matthew_Stallman
1. https://www.facebook.com/people/Richard-Stallman/100000599450702
2. http://www.stallman.org/ ...

➤ #2
Opening page.
➤
\end{verbatim}



\subsection{History}
\label{sec-2.4}


   \emph{M-n} and \emph{M-p} go back and forth in the history


\section{Roadmap}
\label{sec-3}

\subsection{[60\%] v1}
\label{sec-3.1}

\begin{enumerate}
\item $\boxtimes$ Colorize output
\item $\boxtimes$ Async web stuff
\item $\boxtimes$ `M-n' \& `M-p' should work in a context
\item $\boxtimes$ History
\item $\boxtimes$ Some english dictionary
\item $\boxtimes$ Remove useless functions
\item $\Box$ Docstrings
\item $\Box$ Custom functions
\item $\Box$ Must work on Windows
\item $\Box$ Screencast
\end{enumerate}
\subsubsection{[\%] v2}
\label{sec-3.1.1}

\begin{enumerate}
\item $\Box$ Tab-completion specific to the context.
\item $\Box$ More personal productivity commands
\item $\Box$ Wikimedia
\item $\Box$ context in shell
\item $\Box$ Better transmission support
\item $\Box$ Priberam
\item $\Box$ On-the-fly
\item $\Box$ Establish a testing scheme
\item $\Box$ Funding
\end{enumerate}

\end{document}
